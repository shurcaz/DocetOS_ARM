\section{Re-entrant Mutex}
\subsection{Purpose}
When multiple tasks are attempting to access a resource, they are susceptible to race conditions causing unexpected behaviour within the accessed resource. A re-entrant mutex ensures exclusive access to the shared resource in a multi-tasking environment. Safeguarding the resource against race conditions.
\subsection{Specification and Design Considerations}
This mechanism will permit only one task, the owner, to access the resource at a time, preventing interference from other tasks. Being re-entrant will allow the owner task to re-enter the critical section, even if it already holds the mutex, to facilitate nested locking without risking deadlock. This will be particularly valuable in situations where a task invokes a function requiring the same mutex.\hfill\newline
The comparator function created for the mutex waiting list must sort tasks by both priority and time in the waiting list, ensuring the highest priority tasks receives earliest access to the resource and preventing task deadlock.


\subsection{Implemented Design and Functionality}
\subsubsection{Obtaining Mutex}
Before utilizing a resource prone to race conditions, a task can acquire a mutex for that resource. Upon attempting to obtain the mutex, the task checks its lock status. If unclaimed, the task becomes the owner, with an internal counter set to 1. For a task already holding the mutex, the counter increments. If the mutex is owned by another task, the current task is temporarily removed from the scheduler and sent to a wait list, awaiting mutex availability. Once obtained, the task can use the resource exclusively, safeguarding against unintended race conditions.

\subsubsection{Releasing Mutex}
Upon completing the critical section requiring a mutex, the task releases the mutex, decrementing the internal counter. If the counter reaches zero after multiple releases, signalling completion of mutex use, ownership is relinquished. This prompts notification of the wait list, enabling the highest priority waiting task to rejoin the scheduler. This task can then attempt to acquire the mutex and utilize the resource it guards once again.

\subsubsection{Mutex Waiting List Operation}
The Mutex Wait List employs the use of our binary heap module, facilitating notification and removal of only the root task when the mutex becomes available. The heap is organized based on both task priority and the time a task joined the wait list. In the event of two tasks sharing the same priority, the task added earlier takes precedence for notification and release from the wait list. This approach ensures fairness by favouring the task that has been waiting longer when priorities are equal.
